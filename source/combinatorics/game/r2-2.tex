% $date: 2015-12-08
% $timetable:
%   g8r2:
%     2015-12-08:
%       2:

\section*{Игры и стратегии --- 2. Выигрышные и проигрышные позиции}

% $authors:
% - Леонид Андреевич Попов

\begin{problems}

\item
В~кучке лежит $2015$ конфет.
Два игрока по~очереди берут $2$, $4$ или $9$ конфет.
Проигрывает тот, кто не~может сделать ход.
Кто выиграет при правильной игре?

\item
Игра начинается с~числа $2015$.
За~ход разрешается уменьшить имеющееся число на~любой из~его делителей.
Проигрывает тот, кто получит ноль.
Кто выиграет при правильной игре?

\item
На~столе лежит $300$ монет.
За~ход разрешается забрать не~более половины имеющихся монет.
Проигрывает тот, кто не~может забрать хотя~бы одну монету.
Кто выиграет при правильной игре?

\item
Двое ребят играют в~такую игру.
Первый называет число от~$2$ до~$9$, второй умножает его на~любое число
от~$2$ до~$9$, первый умножает результат на~любое число от~$2$ до~$9$ и~т.~д.
Выигрывает тот, у~кого впервые получилось число, большее $1000$.
Кто выиграет при правильной игре?

\item
Игра начинается с~числа $1000$.
За~ход разрешается вычесть из~имеющегося числа любое, не~превосходящее его,
натуральное число, являющееся степенью двойки ($1 = 2^0$).
Выигрывает тот, кто получит ноль.

\item
Двое по~очереди выписывают на~доску натуральные числа от~$1$ до~$1000$.
Первым ходом первый игрок выписывает на~доску число~$1$.
Затем очередным ходом на~доску можно выписать либо число~$2a$, либо
число $a + 1$, если на~доске уже написано число~$a$.
При этом запрещается выписывать числа, которые уже написаны на~доске.
Выигрывает тот, кто выпишет на~доску число $1000$.
Кто выигрывает при правильной игре?

\item
У~дракона есть $2015!$ золотых монет.
Два хоббита по~очереди воруют у~дракона по~$p^n$~монет, где $p$~--- простое
число, а~$n = 1, 2, \ldots$
(например, первый берет $9$~монет, второй $7$, первый $64$ и~т.~д.).
Того, кто не~может ничего украсть, съедает дракон.
Может~ли кто-нибудь из~хоббитов гарантировать свое выживание?

\end{problems}

