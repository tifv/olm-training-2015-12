% $date: 2015-12-09
% $timetable:
%   g8r1:
%     2015-12-09:
%       1:

\section*{Игры и стратегии --- 3}

% $authors:
% - Леонид Андреевич Попов

\begin{problems}

\item
Двое играют в~крестики-нолики на~бесконечной клетчатой бумаге по~таким
правилам: первый ставит два крестика, второй~--- нолик, первый~--- снова два
крестика, второй~--- нолик и~т.~д.
Первый выигрывает, когда на~одной вертикали или горизонтали стоит рядом
$k$~крестиков.
Докажите, что первый всегда может добиться победы, если
\\
\subproblem $k = 6$;
\qquad
\subproblem $k = 100$.

\item
Двое игроков ставят крестики и~нолики на~бесконечной клетчатой бумаге, причем
на~каждый крестик первого игрока второй отвечает 100 ноликами.
Докажите, что первый может добиться, чтобы некоторые четыре крестика образовали
прямоугольник (со~сторонами, параллельными линиям клеток).

\item
Двое игроков отмечают точки плоскости.
Сначала первый отмечает точку красным цветом, затем второй отмечает $100$ точек
синим, затем первый снова одну точку красным, второй $100$ точек синим и~так
далее.
(Перекрашивать уже отмеченные точки нельзя.)
Докажите, что первый может построить правильный треугольник с~красными
вершинами.

\item
Двое по~очереди обводят цветными карандашами стороны клеток на~клетчатой
бумаге.
Первый игрок обводит красным, второй~--- синим.
За~каждый ход можно обвести отрезок между соседними узлами сетки
(составляющий сторону клетки), если этот отрезок еще не~обведен другим игроком.
Докажите, что второй (синий) игрок может помешать первому образовать
красную замкнутую линию.

\end{problems}

