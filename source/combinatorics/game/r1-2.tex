% $date: 2015-12-08
% $timetable:
%   g8r1:
%     2015-12-08:
%       3:

\section*{Игры и стратегии --- 2}
% Империя наносит ответный удар

% $authors:
% - Леонид Андреевич Попов

\begin{problems}

\item
На~листе клетчатой бумаги отмечены $100$ узлов~--- вершины клеток, образующих
квадрат $9 \times 9$.
Два игрока по~очереди соединяют вертикальным или горизонтальным отрезком два
соседних отмеченных узла.
Игрок, после чьего хода образуется один или несколько квадратиков, закрашивает
их в~свой цвет.
Выигрывает тот, кто закрасил больше квадратиков.
Кто выиграет при правильной игре?

\item
Есть $50$~карточек, на~них написаны числа от~$1$ до~$50$, каждое по~одному
разу.
Костя и~Виталик по~очереди берут по~одной карточке.
Костя хочет, чтобы сумма на~его карточках делилась на~$25$.
Сможет~ли Виталик ему помешать,
если
\\
\subproblem Костя ходит вторым;
\qquad
\subproblem Костя ходит первым?

\item
В~крайних клетках полоски $1 \times 103$ стоит по~фишке.
Саша и~Паша ходят по~очереди: за~ход можно сдвинуть свою фишку вправо или влево
на~любое количество клеток от~$1$ до~$4$, но~нельзя перепрыгивать через фишку
противника и~ставить две фишки на~одну клетку.
Проигрывает тот, кто не~может сделать ход.
Первым ходит Саша.
Кто выигрывает при правильной игре?

\item
На~доске записаны два числа: $2014$ и~$2015$.
Петя и~Вася ходят по~очереди, начинает Петя.
За~один ход можно
\\
\textit{(1)}
либо уменьшить одно из~чисел на~его ненулевую цифру или на~ненулевую цифру
другого числа;
\\
\textit{(2)}
либо разделить одно из~чисел пополам, если оно четное.
\\
Выигрывает тот, кто первым напишет однозначное число.
Кто из~них может выиграть, как~бы ни~играл соперник?

\item
На~доске написано число $10^{2015}$.
Двое играют в~следующую игру.
За~один ход с~доски можно стереть два одинаковых числа, либо стереть число~$n$
и~вместо него записать два числа, в~произведение дающих $n$, но~меньших него.
Проигрывает тот, кто не~может сделать ход.
Кто выиграет при правильной игре?

\item
Почтальон Печкин не~хотел отдавать посылку.
Тогда Матроскин предложил ему сыграть в~следующую игру: каждым ходом Печкин
пишет в~строку слева направо буквы \textsf{М} и~\textsf{П}, пока в~строке
не~будет всего $11$~букв.
Матроскин после каждого его хода, если хочет, меняет местами любые две буквы.
Если в~итоге окажется, что записанное <<слово>> является палиндромом
(то~есть одинаково читается слева направо и~справа налево), то~Печкин отдает
посылку.
Сможет~ли Матроскин играть так, чтобы обязательно получить посылку?

\item
На~доске нарисован правильный $108$-угольник.
Двое по~очереди закрашивают его вершины.
Проигрывает тот, после чьего хода несколько закрашенных вершин образуют
правильный многоугольник.
Кто выиграет при правильной игре?

\end{problems}

