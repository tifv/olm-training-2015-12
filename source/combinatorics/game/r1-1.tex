% $date: 2015-12-07
% $timetable:
%   g8r1:
%     2015-12-07:
%       1:

\section*{Игры и стратегии --- 1}

% $authors:
% - Леонид Андреевич Попов

\begin{problems}

\item
По~кругу лежат несколько мандаринок.
Петя и~Вася едят по~очереди одну или две соседние мандаринки;
кто не~может ничего съесть, тот проиграл.
Кто выиграет при правильной игре?

\item
\subproblem
На~доске $16 \times 16$ стоит ферзь.
Его можно двигать либо вправо, либо вверх, либо вверх-вправо.
Проигрывает тот, кто не~может ходить.
При каких начальных положениях ферзя второй игрок победит?
\\
\subproblem
В~двух кучках $n$ и~$k$ камней соответственно.
За~ход разрешается взять либо несколько камней из~одной кучки, либо поровну
из~обеих кучек.
Проигрывает тот, кто не~может сделать ход.
При каких $n$ и~$k$, не~превосходящих $15$, у~первого игрока нет выигрышной
стратегии?

\item
На~доске написано число $12345$.
За~ход разрешается вычесть из~написанного числа любую его ненулевую цифру.
Выигрывает тот, после чьего хода на~доске будет написан ноль.
Кто выиграет при правильной игре?

\item
На~столе донышками вниз стоит $2015$ пустых стаканов.
Два игрока по~очереди переворачивают стаканы, в~том числе и~перевернутые ранее,
по~следующим правилам: за~первый ход можно  перевернуть не~более одного
стакана, за~второй~--- не~более двух и~т.~д.
При этом за~каждый ход необходимо перевернуть хотя~бы один стакан.
Выигрывает тот, после хода которого все стаканы будут расположены донышками
вверх.
Кто может выиграть в~этой игре независимо от~ходов соперника?

\item
Дана шоколадка $700 \times 2015$ ($700$~--- высота, $2015$~--- ширина).
Два человека играют в~следующую игру.
Ход состоит в~том, что можно взять любой отдельный кусок шоколадки (в~начале
игры такой кусок всего один) и~выгрызть из~него кусок в~форме прямоугольника,
причем первому разрешается съедать только прямоугольники, у~которых высота
больше либо равна ширине, а~второму~--- меньше либо равна ширине.
Выигрывает тот, кто доест последний кусочек.
Кто выигрывает при правильной игре?

\item
\emph{Уголком размера $n \times m$,} где $m, n \geq 2$, называется фигура, получаемая
из~прямоугольника размера $n \times m$ клеток удалением прямоугольника размера
$(n - 1) \times (m - 1)$ клеток.
Два игрока по~очереди делают ходы, заключающиеся в~закрашивании в~уголке
произвольного ненулевого количества клеток, образующих прямоугольник или
квадрат.
Пропускать ход или красить клетки дважды нельзя.
Проигрывает тот, после чьего хода все клетки уголка окажутся окрашенными.
Кто из~игроков победит при правильной игре?

\item
У~дракона есть $2015!$ золотых монет.
Два хоббита по~очереди воруют у~дракона по $p^n$~монет, где $p$~--- простое
число, а~$n = 1, 2, \ldots$
(например, первый берет $9$~монет, второй $7$, первый $64$ и~т.~д.).
Того, кто не~может ничего украсть, съедает дракон.
Может~ли кто-нибудь из~хоббитов гарантировать свое выживание?

\item
Двое играют в~следующую игру.
Есть последовательность из~$n$ крестиков и~ноликов.
За~один ход разрешается взять любые $k$ ($k = 1, \ldots, n$) подряд идущих
знаков таких, что эта последовательность начинается с~крестика, а~все остальные
знаки в~ней~--- нолики (допускается последовательность из~одного крестика),
и~инвертировать ее (заменить крестик на~нолик и~нолики на~крестики).
Проигрывает тот, кто не~может сделать ход.
Кто выигрывает при правильной игре (в~зависимости от~начальной позиции)?

\end{problems}

