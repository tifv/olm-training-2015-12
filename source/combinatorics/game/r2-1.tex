% $date: 2015-12-07
% $timetable:
%   g8r2:
%     2015-12-07:
%       2:

\section*{Игры и стратегии --- 1}

% $authors:
% - Леонид Андреевич Попов

\begin{problems}

\item
\subproblem
В~ряд лежат несколько мандаринок.
Петя и~Вася едят по~очереди одну или две соседние мандаринки;
кто не~может ничего съесть, тот проиграл.
Кто выиграет при правильной игре?
\\
\subproblem
Кто победит, если мандаринки лежат по~кругу?

\item
\subproblem
На~столе лежат $125$ спичек.
За~ход разрешается взять $1$, $2$, $3$ или $4$ спички.
Кто не~может сделать ход, проигрывает.
Кто выиграет при правильной игре?
\\
\subproblem
Кто выиграет, если за~ход разрешается взять $1$, $2$, $3$, $4$ или $7$ спичек?
\\
\subproblem
А~если разрешается взять $1$, $2$ или $4$?

\item
На~доске написано число $12345$.
За~ход разрешается вычесть из~написанного числа любую его ненулевую цифру.
Выигрывает тот, после чьего хода на~доске будет написан ноль.
Кто выиграет при правильной игре.

\item
На~клетчатой доске $4 \times 4$ играют двое.
Ходят по~очереди, и~каждый играющий своим ходом закрашивает одну клетку.
Проигрывает тот, после чьего хода образуется квадрат $2 \times 2$, состоящий
из~закрашенных клеток.
Кто выигрывает при правильной игре?

\item
На~столе донышками вниз стоит $2015$ пустых стаканов.
Два игрока по~очереди переворачивают стаканы, в~том числе и~перевернутые ранее,
по~следующим правилам: за~первый ход можно  перевернуть не~более одного
стакана, за~второй~--- не~более двух и~т.~д.
При этом за~каждый ход необходимо перевернуть хотя~бы один стакан.
Выигрывает тот, после хода которого все стаканы будут расположены донышками
вверх.
Кто может выиграть в~этой игре независимо от~ходов соперника?

\item
\subproblem
На~доске $16 \times 16$ стоит ферзь.
Его можно двигать либо вправо, либо вверх, либо вверх-вправо.
Проигрывает тот, кто не~может ходить.
При каких начальных положениях ферзя второй игрок победит?
\\
\subproblem
В~двух кучках $n$ и~$k$ камней соответственно.
За~ход разрешается взять либо несколько камней из~одной кучки, либо поровну
из~обеих кучек.
Проигрывает тот, кто не~может сделать ход.
При каких $n$ и~$k$, не~превосходящих $15$, у~первого игрока нет выигрышной
стратегии?

\item
Дана шоколадка $700 \times 2015$ ($700$~--- высота, $2015$~--- ширина).
Два человека играют в~следующую игру.
Ход состоит в~том, что можно взять любой отдельный кусок шоколадки (в~начале
игры такой кусок всего один) и~выгрызть из~него кусок в~форме прямоугольника,
причем первому разрешается съедать только прямоугольники, у~которых высота
больше либо равна ширине, а~второму~--- меньше либо равна ширине.
Выигрывает тот, кто доест последний кусочек.
Кто выигрывает при правильной игре?

\end{problems}

