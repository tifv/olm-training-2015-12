% $date: 2015-12-05
% $timetable:
%   g8r1:
%     2015-12-05:
%       3:

\section*{Графы. Связность}

% $authors:
% - Глеб Александрович Погудин


\subsection*{Остовные деревья}

\begin{problems}

\item
Может~ли у~графа быть ровно два остовных дерева?

\item
Докажите, что из~любого связного графа можно выкинуть вершину так, что он
не~потеряет связности.

\item
В~стране $n$~городов, между некоторыми есть дороги.
Известно, что из~каждого города можно попасть в~каждый, причем из~каждого
города выходит не~более $d$~дорог.
Докажите, что всю страну можно разделить на~два региона так, что в~каждом
регионе можно будет из~любого города попасть в~любой и~размер каждого региона
будет не~меньше $\bigl \lfloor \frac{n - 1}{d} \bigr \rfloor$.

\end{problems}


\subsection*{А он ещё губку придумал}

\begin{problems}

\item
Пусть дан граф, который при выкидывании любой вершины граф остается связным.
\\
\subproblem
Докажите, что через каждую вершину проходит несамопересекающийся цикл.
\\
\subproblem
Из~всех несамопересекающихся циклов, проходящих через вершину~$u$, выберем тот,
до~которого можно дойти из~вершины~$v$, пройдя по~минимальному возможному числу
ребер.
Докажите, что $v$ просто лежит на~этом цикле.

\item
Пусть при выкидывании любых $k$~вершин граф остается связным.
Докажите, что для любых $k + 1$ вершин найдется несамопересекающийся цикл,
содержащий все эти вершины.

\end{problems}


\subsection*{Задача-аукцион}

\begin{problems}

\item
Оцените сверху количество связных графов на~$10$ вершинах.

\end{problems}

