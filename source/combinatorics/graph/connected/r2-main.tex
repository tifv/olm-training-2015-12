% $date: 2015-12-05
% $timetable:
%   g8r2:
%     2015-12-05:
%       2:

\section*{Графы. Связность}

% $authors:
% - Глеб Александрович Погудин


\subsection*{Остовные деревья}

\begin{problems}

\item
Докажите, что из~любого связного графа можно выкинуть несколько ребер так,
чтобы получилось дерево (такое дерево называется \emph{остовным}).

\item
Может~ли у~графа быть ровно два остовных дерева?

\item
Докажите, что из~любого связного графа можно выкинуть вершину так, что он
не~потеряет связности.

\end{problems}


\subsection*{Связность}

\begin{problems}

\setproblem{4}% XXX

\item
В~графе на~$n$ вершинах степень каждой вершины не~меньше $n / 2$.
Докажите, что он связен.

\item
В~графе на~$n$ вершинах каждая соединена с~каждой (такой граф обозначается
через $K_n$).
Какое наименьшее число ребер нужно удалить, чтобы граф стал несвязным?

\item
В~графе есть две доли~--- из~$n$ и~из~$m$ вершин.
Все вершины из~разных долей соединены друг с~другом, из~одной доли~--- нет
(такой граф обозначается через $K_{m,n}$).
Какое наименьшее число ребер нужно удалить, чтобы граф стал несвязен?

\item
Докажите, что если в~несвязном графе все ребра заменить на~неребра, а~неребра
на~ребра (то есть соединить все несмежные вершины и разъединить смежные),
то~получится связный граф.

\end{problems}


\subsection*{Задача-аукцион}

\begin{problems}

\item
Оцените снизу количество связных графов на~$10$ вершинах.

\end{problems}

