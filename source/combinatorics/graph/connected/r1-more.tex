% $date: 2015-12-06
% $timetable:
%   g8r1:
%     2015-12-06:
%       2:

% $caption: Графы. Связность. Добавка

% $matter[-contained,no-header]:
% - verbatim: \section*{Графы. Связность}
% - verbatim: \setproblem{6}
% - .[contained]

\subsection*{Добавка}

% $authors:
% - Глеб Александрович Погудин

\begin{problems}

\item
В~стране несколько городов, некоторые пары городов соединены беспосадочными
рейсами одной из~$n$ авиакомпаний, причем из~каждого города есть ровно
по~одному рейсу каждой из~авиакомпаний.
Известно, что из~каждого города можно долететь до~любого другого (возможно,
с~пересадками).
Из-за финансового кризиса были закрыты $(n - 1)$ рейсов, но~ни~в~одной
из~авиакомпаний не~закрыли более одного рейса.
Докажите, что по-прежнему из~каждого города можно долететь до~любого другого.

\item
Обозначим через $C_n$ количество связных графов на~$n$ вершинах.
Придумайте алгоритм, вычисляющий $C_n$ за~$O(n^2)$.

\end{problems}

