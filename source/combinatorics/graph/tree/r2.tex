% $date: 2015-12-04
% $timetable:
%   g8r2:
%     2015-12-04:
%       2:

\section*{Графы. Деревья}

% $authors:
% - Антон Сергеевич Гусев

\begin{problems}

\item
\subproblem
Пусть в~графе нет циклов и~вершин степени ноль.
Докажите, что есть вершина степени один.
Такая вершина, кстати, еще называется \emph{висячей.}
\\
\subproblem
Докажите, что их на~самом деле хотя~бы две.
\\
\subproblem
Пусть в~графе есть вершина степени~$d$.
Докажите, что висячих вершин на~самом деле хотя~бы $d$.

\end{problems}

\definition
Связный граф без циклов называется \emph{деревом.}

\begin{problems}

\item
Докажите, что в~дереве с~$n$ вершинами всегда ровно $(n - 1)$ ребер.

\item
Каким может быть количество висячих вершин в~дереве на~$n$ вершинах?

\item
Можно~ли раскрасить ребра куба в~два цвета так, чтобы по~ребрам каждого цвета
можно было пройти из~любой вершины в~любую?

\item
Туристическая фирма провела акцию:
<<Купи путевку в~Египет, приведи четырех друзей, которые также купят путевку,
и~получи стоимость путевки обратно>>.
За~время действия акции $13$~покупателей пришли сами, остальных привели друзья.
Некоторые из~них привели ровно по~$4$ новых клиента, а~остальные $100$
не~привели никого.
Сколько туристов отправились в~Страну Пирамид бесплатно?

\item
В~стране 15 городов, некоторые из~них соединены авиалиниями, принадлежащими
трем авиакомпаниям.
Известно, что даже если любая из~авиакомпаний прекратит полеты, можно будет
добраться из~любого города в~любой другой (возможно, с~пересадками), пользуясь
рейсами оставшихся двух компаний.
Какое наименьшее количество авиалиний может быть в~стране?

\item
Докажите, что все следующие утверждения эквивалентны:
\\
\subproblemy{1}
граф~$G$~--- дерево;
\\
\subproblemy{2}
любые две вершины графа~$G$ соединены единственным простым путем;
\\
\subproblemy{3}
граф~$G$ связен и~$p = q + 1$, где $p$~--- количество вершин, а~$q$~---
количество ребер;
\\
\subproblemy{4}
граф~$G$~--- ацикличен (не~содержит циклов), и~$p = q + 1$, где $p$~---
количество вершин, а~$q$~--- количество ребер;
\\
\subproblemy{5}
граф~$G$~--- ацикличен, и~при добавлении любого ребра для несмежных вершин
появляется один простой цикл;
\\
\subproblemy{6}
граф~$G$~--- связный граф, отличный от~$K_p$ для $p \geq 3$, а~также при
добавлении любого ребра для несмежных вершин появляется один простой цикл;
\\
\subproblemy{7}
граф~$G$~--- граф, отличный от~$K_3 \cup K_1$ и~$K_3 \cup K_2$,
а~также $p = q + 1$, где $p$~--- количество вершин, а~$q$~--- количество ребер,
и~при добавлении любого ребра для несмежных вершин появляется один простой
цикл.

\end{problems}

