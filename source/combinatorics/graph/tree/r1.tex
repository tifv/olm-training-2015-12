% $date: 2015-12-04
% $timetable:
%   g8r1:
%     2015-12-04:
%       1:

\section*{Графы. Деревья}

% $authors:
% - Антон Сергеевич Гусев

\definition
Связный граф без циклов называется \emph{деревом}.

\begin{problems}

\item
Можно~ли раскрасить ребра куба в~два цвета так, чтобы по~ребрам каждого цвета
можно было пройти из~любой вершины в~любую?

\item
В~стране 15 городов, некоторые из~них соединены авиалиниями, принадлежащими
трем авиакомпаниям.
Известно, что даже если любая из~авиакомпаний прекратит полеты, можно будет
добраться из~любого города в~любой другой (возможно, с~пересадками), пользуясь
рейсами оставшихся двух компаний.
Какое наименьшее количество авиалиний может быть в~стране?

\item
Докажите, что все следующие утверждения эквивалентны:
\\
\subproblemy{1}
граф~$G$~--- дерево;
\\
\subproblemy{2}
любые две вершины графа~$G$ соединены единственным простым путем;
\\
\subproblemy{3}
граф~$G$ связен и~$p = q + 1$, где $p$~--- количество вершин, а~$q$~---
количество ребер;
\\
\subproblemy{4}
граф~$G$~--- ацикличен (не~содержит циклов), и~$p = q + 1$, где $p$~---
количество вершин, а~$q$~--- количество ребер;
\\
\subproblemy{5}
граф~$G$~--- ацикличен, и~при добавлении любого ребра для несмежных вершин
появляется один простой цикл;
\\
\subproblemy{6}
граф~$G$~--- связный граф, отличный от~$K_p$ для $p \geq 3$, а~также при
добавлении любого ребра для несмежных вершин появляется один простой цикл;
\\
\subproblemy{7}
граф~$G$~--- граф, отличный от~$K_3 \cup K_1$ и~$K_3 \cup K_2$,
а~также $p = q + 1$, где $p$~--- количество вершин, а~$q$~--- количество ребер,
и~при добавлении любого ребра для несмежных вершин появляется один простой
цикл.

\item
Докажите, что в~графе можно раскрасить не~более половины вершин так, чтобы
любая нераскрашенная вершина была соединена ребром с~одной из~раскрашенных.

\item
В~стране $100$ городов, из~каждого города выходит хотя~бы одна дорога.
Докажите, что можно закрыть несколько дорог так, чтобы по-прежнему из~каждого
города выходило не~менее одной дороги и~при этом по~крайней мере
из~$67$ городов выходило ровно по~одной дороге.

\item
Пусть граф~$G$ таков, что при выкидывании любой вершины он остается связным.
Докажите, что в~нем между любыми двумя вершинами существует хотя~бы два
непересекающихся пути.

\item
\emph{<<Пятнашки на~графе>>.}
Дан двусвязный граф (при выкидывании любой вершины он остается связным).
В~нем есть треугольник (цикл длины три).
Одна вершина пустая, а~в~остальных расставлены различные фишки.
Фишки можно перемещать на~смежную пустую вершину.
Докажите, что можно добиться произвольной расстановки фишек.

\end{problems}

