% $date: 2015-12-04
% $timetable:
%   g8r2:
%     2015-12-04:
%       1:

\section*{НОД и НОК}

% $authors:
% - Владимир Викторович Трушков

\begin{problems}

\item
Найдите $(451, 287)$; $(1381955, 690713)$.

\item
Сколько существует пар натуральных чисел, у~которых наименьшее общее кратное
(НОК) равно $2000$?

\item
Найдите наибольший общий делитель чисел $2 n + 13$ и~$n + 7$.

\item
Докажите, что дробь\enspace
$\dfrac{12 n + 1}{30 n + 2}$\enspace
несократима ни~при каком натуральном $n$.

\item
Найдите $(2^{100} - 1, 2^{120} - 1)$.

\item
Докажите, что среди чисел
$10^{100} + 1$, $10^{101} + 1$, $10^{102} + 1$, $10^{103} + 1$ найдется число,
взаимно простое с~остальными тремя.

\item
Докажите, что для любых натуральных чисел $a$ и~$b$ верно равенство
$(a, b) \cdot [a, b] = a b$.

\item
Докажите, что если $[a, a + 5] = [b, b + 5]$ ($a$, $b$~--- натуральные),
то~$a = b$.

\item
Докажите, что $(b c, c a, a b)$ делится на~$(a, b, c)^2$.

\item
Вася выписал на~доске $100$ чисел меньших, чем сотое по~счету простое число.
Докажите, что какое-то из~выписанных чисел является делителем произведения
остальных $99$.

\item
Даны натуральные числа $a$ и~$b$.
Известно, что $a^2 + b^2$ делится на~$ab$.
Докажите, что $a = b$.

\item
Известно, что $(m, n) = 1$.
Каково наибольшее возможное значение $(m + 2000 n, n + 2000 m)$?

\end{problems}

