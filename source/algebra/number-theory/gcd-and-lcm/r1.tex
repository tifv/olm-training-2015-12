% $date: 2015-12-04
% $timetable:
%   g8r1:
%     2015-12-04:
%       3:

\section*{НОД и НОК}

% $authors:
% - Антон Сергеевич Гусев

\begin{problems}

\item
Сколько существует пар натуральных чисел, у~которых НОК равно $2000$?

\item
Докажите, что дробь\enspace
$\dfrac{12 n + 1}{30 n + 2}$\enspace
несократима при всех натуральных~$n$.

\item
Найдите $(2^{100} - 1, 2^{120} - 1)$.

\item
Докажите, что среди чисел
$10^{100} + 1$, $10^{101} + 1$, $10^{102} + 1$, $10^{103} + 1$ найдется число,
взаимно простое с~остальными тремя.

\item
Докажите, что для любых натуральных чисел $a$ и~$b$ верно равенство
$(a, b) \cdot [a, b] = a b$.

\item
Докажите, что
\(
    a b c
=
    [a, b, c] \cdot (a b, b c, c a)
=
    [a b, b c, c a] \cdot (a, b, c)
\)
для любых натуральных $a$, $b$, $c$.

\item
Докажите, что если $[a, a + 5] = [b, b + 5]$ ($a$, $b$~--- натуральные),
то~$a = b$.

\item
Вася выписал на~доске $100$ чисел меньших, чем сотое по~счету простое число.
Докажите, что какое-то из~выписанных чисел является делителем произведения
остальных $99$.

\item
Даны натуральные числа $a$ и~$b$.
Известно, что $a^2 + b^2$ делится на~$ab$.
Докажите, что $a = b$.

\item
Известно, что $(m, n) = 1$.
Каково наибольшее возможное значение $(m + 2000 n, n + 2000 m)$?

\item
Найдите все натуральные~$n$, для которых $n 2^{n-1} + 1$ является точным
квадратом.

\end{problems}

