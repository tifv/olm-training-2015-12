% $date: 2015-12-03
% $timetable:
%   g8r2:
%     2015-12-03:
%       2:

\section*{Числа. Простые и не очень}

% $authors:
% - Владимир Викторович Трушков

\begin{problems}

\item
Делится~ли число
\(
    \underbrace{66 {\ldots} 6}_{\text{$2007$ цифр}}
\)
на~$9$?

\item
Простым или составным является число $4^{9} + 6^{10} + 3^{20}$?

\item
Найдите все натуральные~$n$, для которых число $n^5 + n + 1$ является простым.

\item
Найдите все целые $a$ и~$b$ такие, что $a^4 + 4 b^4$ является простым числом.

\item
Известно, что $a = 3^{2004} + 2$.
Верно~ли, что $a^2 + 2$~--- простое число?

\item
Докажите, что число $1998 \cdot 2000 \cdot 2002 \cdot 2004 + 16$ является
квадратом натурального числа.

\item
Докажите, что число
\(
    \underbrace{11 {\ldots} 1}_{2 \cdot 962}
    -
    \underbrace{22 {\ldots} 2}_{962}
\)
является квадратом некоторого натурального числа.

\item
Сравните числа: $99!$ и~$50^{99}$.

\item
При каком наименьшем $n$ число
\(
    \underbrace{22 {\ldots} 2}_{\text{$n$ цифр}}
\)
кратно $17$?

\item
Сравните дроби:\enspace
$\dfrac{2006}{2007}$\enspace
и\enspace
$\dfrac{20062006}{20072007}$.

\item
Докажите, что\enspace
%$1 / 10 + 1 / 11 + 1 / 12 + \ldots + 1 / 100 > 1$.
\(
    \dfrac{1}{10} + \dfrac{1}{11} + \dfrac{1}{12}
    + \ldots +
    \dfrac{1}{100}
>
    1
\).

\item
Докажите, что\enspace
\(
    \dfrac{1}{2} - \dfrac{1}{3} + \dfrac{1}{4} - \dfrac{1}{5}
    + \ldots +
    \dfrac{1}{98} - \dfrac{1}{99} + \dfrac{1}{100}
>
    \dfrac{1}{5}
\).

\end{problems}

