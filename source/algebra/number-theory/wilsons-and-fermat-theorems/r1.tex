% $date: 2015-12-05
% $timetable:
%   g8r1:
%     2015-12-05:
%       2:

\section*{Вильсон, Ферма и другие}

% $authors:
% - Владимир Викторович Трушков

\begin{problems}

\item
Докажите, что $p$~--- простое тогда и~только тогда, когда
$(p - 2)! \equiv 1 \pmod{p}$.

\item
Докажите, что $(2 p - 1)! - p$ делится на~$p^2$.

\item
Докажите, что числа $p$ и~$p + 2$ являются простыми числами-близницами
тогда и~только тогда, когда
\(
    4 \bigl( (p - 1)! + 1 \bigr) + p \equiv 0 \pmod{p^2 + 2 p}
\).

\item
При каких $k$ число $C_{p}^{k}$ делится на~$p$ при простом~$p$?

\item
Докажите, что если $n$~--- составное, то~хотя~бы один из~биномиальных
коэффициентов $C_{n}^{1}, C_{n}^{2}, \ldots, C_{n}^{n-1}$ не~кратен $n$.

\item
Докажите малую теорему Ферма по~индукции (если $(a^p - a)$ делится на~$p$,
то~$(a + 1)^p - (a + 1)$ делится на~$p$).

\item
Пусть $p > 2$~--- простое число.
Сколькими способами можно провести через вершины правильного $p$-угольника
замкнутую ориентированную $p$-звенную ломаную?
(Ломаные, которые можно совместить поворотом, считаются одинаковыми).
Найдите формулу и~выведите из~нее теорему Вильсона.

\item
%\textit{Лемма об уточнении показателя.}
Докажите, что если $a \equiv b \pmod{p}$, то~$a^p \equiv b^p \pmod{p^2}$.

\item
Для натурального числа~$n$ обозначим все его делители:
$n = d_1 > d_2 > \ldots > d_k = 1$.
Докажите, что $d_1 - d_2 + d_3 - \ldots + (-1)^{k - 1} d_k = n - 1$
тогда и~только тогда, когда $n$~--- простое или $n=4$.

\item
Для натурального числа $n$ обозначим все его делители:
$n = d_1 > d_2 > \ldots > d_k = 1$.
Докажите, что существует такое $n > 10^{1000}$, что
$d_1 - d_2 + d_3 - \ldots + (-1)^{k-1} d_k = 2 n / 3$.

\item
Докажите, что существует бесконечно много составных чисел вида $10^n + 3$.

\item
Найдите все числа, взаимно простые с~каждым из~чисел вида
$a_n = 2^n + 3^n + 6^n - 1$.

\end{problems}

