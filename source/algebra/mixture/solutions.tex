% $groups$delegate: false
% $groups$delegate$into: false
% $groups$matter: false
% $groups$matter$into: false

\section*{УТЮМовский разнобой. Решения}

% $authors:
% - Владимир Викторович Трушков

\begin{problems}

\itemy{1/1}
Для различных натуральных чисел $a$ и~$b$ докажите неравенство\enspace
\[
    a^4 + b^4
\geq
    2 a^2 b^2 + 4 a b + 1
\, . \]

 \textit{Подсказка.} Сделайте замену $b = a + x$.

\itemy{2/-}
Найдите такое наименьшее натуральное число~$n$, что в~каждом наборе, состоящем
из~$n$ натуральных чисел, найдутся три различных числа $a$, $b$ и~$c$, для
которых $a b + b c + c a$ делится на~3.

 \textit{Ответ:} $6$.
 \textit{Решение.}
 Недолгий перебор показывает, что $a b + b c + c a$ делится на~$3$ тогда
 и~только тогда, когда все остатки от~деления чисел $a$, $b$ и~$c$ на~$3$
 одинаковы, либо хотя~бы два из~этих чисел делятся на~$3$.
 Из~чисел $1$, $2$, $3$, $4$, $5$ такую тройку выбрать нельзя, так что чисел
 должно быть хотя~бы шесть.
 Возьмем любые 6 натуральных чисел.
 Если среди них нет двух чисел, делящихся на~$3$, то~хотя~бы у~пяти из~них
 остатки от~деления на~$3$ равны $1$ или $2$.
 Но~тогда найдутся три числа с~одинаковыми остатками.

\itemy{3/2}
При каких натуральных~$n$ число $4^{n} + 6^{n} + 9^{n}$ является точным
квадратом?

 \textit{Ответ:} ни~при каких.
 \textit{Решение.}
 \(
     4^{n} + 6^{n} + 9^{n} = a^2
 \quad\Rightarrow\quad
     (2^{n} + 3^{n})^2 - 6^{n} = a^2
 \quad\Rightarrow\quad
     6 n = (2 n + 3 n - a) (2 n + 3 n + a)
 \).
 Так как $2^{n} + 3^{n}$ не~делится ни~на~$2$, ни~на~$3$, $a$ тоже не~делится
 ни~на~$2$, ни~на~$3$.
 Поэтому ровно одна из~этих скобок делится на~$3$, а~значит, и~на~$3^{n}$.
 Тогда это должна быть б\'{о}льшая скобка.
 Значит, $2^{n} + 3^{n} + a$ делится на~$3^{n}$.
 Обе эти скобки четные, но~на~$4$ делится только одна из~них, так как их
 разность $2 a$ на~$4$ не~делится.
 Разберем два случая.
 \\
 \textit{(1)}
 Вторая скобка не~делится на~$4$.
 Тогда имеем $2^{n} + 3^{n} - a = 2^{n-1}$
 и~$2^{n} + 3^{n} + a = 2 \cdot 3^{n}$.
 Из~первого следует, что $a > 3 n$, а~из~второго следует, что $a < 3 n$.
 Поэтому этот случай невозможен.
 \\
 \textit{(2)}
 Первая скобка не~делится на~$4$.
 Тогда она вообще равна $2$ и~$a = 2^{n} + 3^{n} - 2$.
 Тогда вторая скобка будет равна $2 (2^{n} + 3^{n} - 1)$, и~это должно быть
 равно $2^{n-1} \cdot 3^{n}$.
 Пусть $x = 2^{n}$, $y = 3^{n}$, тогда
 \(
     2 (x + y - 1) = x y / 2
 \quad\Rightarrow\quad
     x y = 4 x + 4 y - 4
 \quad\Rightarrow\quad
     (x - 4) (y - 4) = 12
     ?
     (3^{n} - 4) (2^{n} - 4) = 12
 \).
 Чтобы левая часть делилась на~$4$, $n$ должно быть больше $1$.
 Но~если $n > 2$, то~$3^{n} - 4 > 12$, а~вариант $n = 2$, как легко убедиться,
 тоже не~подходит.

\itemy{4/-}
Существует~ли такое натуральное число~$n$, что $n^2 + S(n) = 99{\ldots}98$
($99$~девяток), где через $S(n)$ обозначена сумма цифр числа~$n$? 

 \textit{Ответ:} не~существует.
 \textit{Решение.}
 Как известно, число и~сумма его цифр имеют одинаковые остатки от~деления
 на~$9$.
 Перебирая все возможные остатки $0, 1, \ldots, 8$ от~деления числа $n$ на~$9$,
 убеждаемся, что ни~в~одном из~случаев сумма $n^{2} + S(n)$ не~дает остатка 8,
 как число $99{\ldots}998$.

\itemy{-/3}
Существует~ли такое натуральное число~$n$, что $n^2 + S(n) + 1 = 2015^{2014}$,
где через $S(n)$ обозначена сумма цифр числа~$n$?

 \textit{Ответ:} не~существует.
 \textit{Решение.}
 Положим $k = 2015^{1007}$.
 Тогда равенство из~условия можно переписать в~виде
 $S(n) + 1 = k^{2} - n^{2} = (k - n) (n + k)$,
 откуда $S(n) \geq n + k - 1 > n$.
 Но, как легко видеть, $S(n) \leq n$ при любом натуральном~$n$.

\itemy{5/4}
Для положительных чисел $a$, $b$, $c$ и~$d$ докажите неравенство
\[
    128 (a + b + c + d)
<
    (4 + a^2) (4 + b^2) (4 + c^2) (4 + d^2)
\, . \]

 \textit{Решение.}
 Если раскрыть скобки в~правой части, то~среди слагаемых будут $256$
 и~$64 (a^{2} + b^{2} + c^{2} + d^{2})$.
 Забудем про все остальные слагаемые в~правой части (они положительны),
 разделим обе части на~$64$ и~применим неравенство $1 + x^{2} > 2 x$.

\itemy{6/-}
В~ряд выписано $1000$ натуральных чисел.
Произведение любых двух соседних~--- куб натурального числа.
Докажите, что произведение двух крайних~--- куб натурального числа.

 \textit{Решение.}
 Если произведение двух чисел~--- куб натурального числа, то~для каждого
 простого числа сумма степеней, с~которыми оно входит в~разложения этих двух
 чисел на~простые множители, делится на~$3$.
 Отсюда следует, что если взять два наших числа, идущие в~ряду через одно,
 то~для каждого простого числа показатели степеней, в~которых оно входит
 в~разложения этих двух чисел на~простые множители, дают одинаковые остатки
 от~деления на~$3$.
 Значит, набор остатков от~деления на~$3$ показателей степеней, в~которых
 простые числа входят в~разложение на~простые множители, один и~тот~же для всех
 чисел, стоящих в~нашем ряду на~четных местах.
 В~частности, у~$1000$-го числа он такой~же, как у~второго, откуда и~следует
 утверждение задачи.

\itemy{7/5}
Найдите все пары натуральных чисел $x$ и~$y$, для которых $x y^2 + 7$ делится
на~$x^2 y + x$.

 \textit{Ответ:} $x = 1, y = 1, 3, 7$; $x = y = 7$.
 \textit{Решение.}
 Если $x y^{2} + 7$ делится на~$x^{2} y + x$, то~$7$ делится на~$x$.
 Значит, $x = 1$ или $x = 7$.
 Если $x = 1$, $y^{2} + 7$ должно делиться на~$y + 1$.
 Так как $y^{2} + 7 = (y + 1) (y - 1) + 8$, в~этом случае необходимо
 и~достаточно, чтобы $8$ делилось на~$y + 1$, откуда $y = 1, 3, 7$.
 Если $x = 7$, $7 y^{2} + 7$ должно делиться на~$49 y + 7$, то~есть
 $y^{2} + 1$ должно делиться на~$7 y + 1$.
 Так как $49 y^{2} + 49 = (7 y + 1) (7 y - 1) + 50$, в~этом случае $50$ должно
 делиться на~$7 y + 1$.
 Проверка показывает, что годится только $y = 7$.

\itemy{8/6}
Найдите наименьшее натуральное~$a$, для которого неравенство
$(n!)^2 \cdot a^n > (2 n)!$ справедливо при всех натуральных~$n$.

 \emph{Ответ:} $a = 4$.
 \emph{Решение.}
 Контрпример для $a = 3$~--- $n = 5$.
 Докажем для $a = 4$.
 Поделив обе части неравенства на~$(n!)^2$, получим неравенство
 $2^{2n} > (2 n)! / (n!)^2$, которое следует из~того, что $2^{2n}$ равно сумме
 всех биномиальных коэффициентов с~основанием $2n$.

\itemy{-/7}
Назовем натуральное число \emph{хорошим,} если оно представимо в~виде суммы
трех натуральных чисел $a < b < c$ таких, что $c$ делится на~$b$ и~$b$
делится на~$a$.
Найдите наибольшее нехорошее число.

 \textit{Ответ:} $24$.
 \textit{Решение.}
 Убедиться в~том, что число $24$~--- нехорошее, можно с~помощью разумно
 организованного перебора, например, такого: если $n = a + b + c$,
 $c$ делится на~$b$ и~$b$ делится на~$a$, то~$a$~--- делитель $n$ и~число
 $n / a - 1 = (b / a) \cdot (c / b + 1)$ составное.
 Но~числа $24 - 1$, $12 - 1$, $8 - 1$, $6 - 1$, $4 - 1$ и~даже $3 - 1$ простые.
 Докажем, что все числа, большие $24$, хорошие.
 Очевидно, кратное хорошего числа тоже хорошее: если $n = a + b + c$,
 то~$k n = k a + k b + k c$.
 Каждое нечетное число, не~меньшее $7$, хорошее: $2 s + 3 = 1 + 2 + 2 s$.
 Четные числа, кратные $5$, хорошие, потому что $10 = 1+3+6$.
 Осталось рассмотреть степени двойки, большие 24, и~числа вида $2^{m} \cdot 3$,
 большие $24$.
 Первые хороши благодаря разложению $16 = 1 + 3 + 12$, а~вторые~--- благодаря
 тому, что $48$ делится на~$16$.

\itemy{-/8}
Целые числа $x$ и~$y$ таковы, что $(x^2 - 2 x y + y^2 - 5 x + 7 y)$
и~$(x^2 - 3 x y + 2 y^2 + x - y)$ делятся на~$17$.
Докажите, что $(x y - 12 x + 15 y)$ также делится на~$17$.

 \emph{Решение.}
 Все сравнения по~модулю~$17$.
 Заметим, что $x^2 - 3 x y + 2 y^2 + x - y = (x - y) (x - 2 y + 1)$, то~есть
 либо $x \equiv y$, либо $x \equiv 2 y - 1$.
 В~первом случае получим $x^2 - 2 x y + y^2 - 5 x + 7 y \equiv 2 x$, то~есть
 $x$ и~$y$ делятся на~$17$, откуда $x y - 12 x + 15 y$ делится на~$17$.
 Во~втором случае
 \(
     x^2 - 2 x y + y^2 - 5 x + 7 y
 \equiv
     (2 y - 1)^2 - 2 (2 y - 1) y + y^2 - 5 (2 y - 1) + 7 y
 \), откуда после преобразований получаем, что $y^2 - 5 y + 6$ делится на~$17$.
 Также,
 \(
     x y - 12 x + 15 y
 \equiv
     2 y^2 - 10 y + 12
 =
     2 (y^2 - 5 y + 6)
 \).
 Значит, и~в~этом случае $x y - 12 x + 15 y$ делится на~$17$.

\itemy{-/9}
Для каждого натурального $n > 1$ обозначим через $d_{n}$ наибольший его
делитель, меньший самого числа~$n$.
Докажите, что для бесконечно многих $n$ число $d_{n} + d_{n+1}$ является
точным квадратом.

 \textit{Решение.}
 Заметим, что $d_{6+420k} + d_{7+420k} = 3 + 210 k + 1 + 60k = 4 + 270 k$.
 Среди чисел вида $4 + 270 k$ содержатся квадраты всех чисел, дающих
 остаток~$2$ при делении на~$270$, значит, квадратов такого вида бесконечно
 много, откуда и~вытекает утверждение задачи.

\end{problems}

