% $groups$delegate: false
% $groups$delegate$into: false
% $groups$matter: false
% $groups$matter$into: false

% $matter[-header,-no-header]:
% - .[no-header]

% $matter[-matter-guard,no-header]:
% - verbatim: \begingroup \let\ifsourcelinks\iftrue
%   condition: source-link
% - .[matter-guard]
% - verbatim: \endgroup % \let\ifsourcelinks
%   condition: source-link

\begingroup
\providecommand\ifsourcelinks{\iffalse}
\providecommand\url{\texttt}

\strut

\vfill

\begin{center}
\scalebox{2}{\(\displaystyle
\int\limits_{\text{3 декабря 2015}}^{\text{10 декабря 2015}}
    \dbinom{\text{\ Московские сборы\ }}{\text{\ секция математики\ }}
    \, \mathrm{d} t
\)}
\end{center}

\vfill

\strut

\clearpage


\subsection*{Немного о~группах}

Занятия проходили в~двух группах:
\begingroup\multicolsep=\parskip
\begin{multicols}{2}
8-2  <<Зяблики>>
\\
8-1  <<Снегири>>
\end{multicols}
\endgroup

Группа \mbox{8-1} была собраны из~предположительно более сильных школьников,
и~задачи там в среднем сложнее, чем в~группе \mbox{8-2}.


\subsection*{Немного о~структуре}

Материалы разбиты по~группам, в~пределах каждой группы отсортированы по~темам:
\begin{itemize}
    \item алгебра;
    \item теория чисел;
    \item многочлены;
    \item неравенства;
\smallskip
    \item геометрия;
\smallskip
    \item комбинаторика;
    \item теория графов.
\end{itemize}


Материалы, общие для нескольких групп, дублируются.
\ifsourcelinks
Все материалы сопровождаются ссылками на~исходные файлы \LaTeX.
\fi

\endgroup % \def\ifsourcelinks \def\url

