% $date: 2015-12-07
% $timetable:
%   g8r2:
%     2015-12-07:
%       3:

% $caption: Поворот

\section*{Поворот}

% $authors:
% - Андрей Юрьевич Кушнир

\begin{problems}

\item
Докажите, что треугольник $ABC$ является равносторонним тогда и~только тогда,
когда вершина~$B$ переходит в~вершину~$C$ при повороте с~центром~$A$ и~углом
$\pm 60^{\circ}$.

\item
При некотором повороте на~угол~$\alpha$ точка~$A$ переходит в~$A'$, а~точка~$B$
переходит в~точку~$B'$.
Докажите, что угол между направленными отрезками $AB$ и~$A'B'$ равен $\alpha$
(все углы в~этой задаче считаются против часовой стрелки).

\item
На~сторонах треугольника $ABC$ внешним образом построены равносторонние
треугольники $A B C_1$, $B C A_1$, $C A B_1$.
Докажите, что $A A_1 = B B_1 = C C_1$.

\item
Внутри квадрата $ABCD$ отмечена точка~$X$.
Докажите, что прямые, проведенные через вершины $B$, $C$, $D$, $A$
перпендикулярно прямым $AX$, $BX$, $CX$, $DX$ соответственно, пересекаются
в~одной точке.

\item
На~отрезке $AC$ отмечена точка~$B$.
Треугольники $ABX$ и~$BCY$~--- равносторонние и~лежат в~одной полуплоскости
относительно $AC$.
Докажите, что точка~$B$ и~середины отрезков $AY$ и~$CX$ служат вершинами
равностороннего треугольника.

\item
Через вершину~$A$ квадрата $ABCD$ внутри угла $BAD$ проведено два луча.
Докажите, что отрезок, соединяющий проекции вершины~$B$ на~эти лучи, равен
и~перпендикулярен отрезку, соединяющему проекции вершины~$D$ на~эти лучи.

\item
Даны три параллельные прямые.
Циркулем и~линейкой отметьте по~точке на~каждой прямой, являющиеся вершинами
равностороннего треугольника.

\item
Дан правильный шестиугольник $ABCDEF$.
Докажите, что точка~$A$ и~середины отрезков $BD$ и~$EF$ являются вершинами
равностороннего треугольника.

\item
На~сторонах $AB$, $AC$ треугольника $ABC$ внешним образом построены квадраты
$ABXP$ и~$ACYQ$.
Докажите, что медиана~$AM$ треугольника $ABC$ перпендикулярна $PQ$.

\item
Даны два поворота с~центрами $A$ и~$B$ и~с~известными углами $\alpha$
и~$\beta$, причем $\alpha + \beta \neq 360^{\circ}$.
\\
\subproblem
Циркулем и~линейкой постройте неподвижную точку композиции этих поворотов.
\\
\subproblem
Докажите, что композицией этих поворотов является поворот
на~угол $\alpha + \beta$.

\end{problems}

