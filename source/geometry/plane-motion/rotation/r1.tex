% $date: 2015-12-07
% $timetable:
%   g8r1:
%     2015-12-07:
%       2:

% $caption: Поворот

\section*{Вот это поворот}

% $authors:
% - Андрей Юрьевич Кушнир

\begin{problems}

\item
Как изометрии, сохраняющие ориентацию, действуют на~множестве векторов
плоскости?
Ответив на~этот вопрос и~вспомнив теорему Шаля, определите, каким
преобразованием плоскости является композиция
\\
\subproblem
двух параллельных переносов на~векторы $\mathbf{u}$ и~$\mathbf{v}$;
\\
\subproblem
параллельного переноса на~вектор $\mathbf{u}$ и~поворота на~угол~$\alpha$;
\\
\subproblem
двух поворотов с~углами $\alpha$ и~$\beta$?
\\
Местоположение центров поворотов неизвестно, если ответом к~одному из~пунктов
служит поворот, то~его центр находить не~требуется.

\item
На~сторонах треугольника $ABC$ внешним образом построены равносторонние
треугольники $A B C_1$, $B C A_1$, $C A B_1$.
Докажите, что $A A_1 = B B_1 = C C_1$.

\item
Внутри квадрата $ABCD$ отмечена точка~$X$.
Докажите, что прямые, проведенные через вершины $B$, $C$, $D$, $A$
перпендикулярно прямым $AX$, $BX$, $CX$, $DX$ соответственно, пересекаются
в~одной точке.

\item
На~отрезке $AC$ отмечена точка~$B$.
Треугольники $ABX$ и~$BCY$~--- равносторонние и~лежат в~одной полуплоскости
относительно $AC$.
Докажите, что точка~$B$ и~середины отрезков $AY$ и~$CX$ служат вершинами
равностороннего треугольника.

\item
Через вершину~$A$ квадрата $ABCD$ внутри угла $BAD$ проведено два луча.
Докажите, что отрезок, соединяющий проекции вершины~$B$ на~эти лучи, равен
и~перпендикулярен отрезку, соединяющему проекции вершины~$D$ на~эти лучи.

\item
Дан правильный шестиугольник $ABCDEF$.
Докажите, что точка~$A$ и~середины отрезков $BD$ и~$EF$ являются вершинами
равностороннего треугольника.

\item
На~продолжениях высот, опущенных из~вершин $B$, $C$ остроугольного треугольника
$ABC$, отмечены точки $B_1$, $C_1$ соответственно, так что
$B B_1 = AC$, $C C_1 = AB$.
Докажите, что отрезки $A B_1$ и~$A C_1$ перпендикулярны и~равны.

\item
На~сторонах $AB$, $AC$ треугольника $ABC$ внешним образом построены квадраты
$ABXP$ и~$ACYQ$.
Докажите, что медиана~$AM$ треугольника $ABC$ перпендикулярна $PQ$.

\item
Даны два поворота с~центрами $A$ и~$B$ и~с~известными углами $\alpha$
и~$\beta$, причем $\alpha + \beta \neq 360^{\circ}$.
Циркулем и~линейкой постройте неподвижную точку композиции этих поворотов.

\end{problems}

