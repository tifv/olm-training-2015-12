% $date: 2015-12-05
% $timetable:
%   g8r2:
%     2015-12-05:
%       3:

\section*{Параллельный перенос}

% $authors:
% - Андрей Юрьевич Кушнир

\begin{problems}

\item
Два селения (точки на~плоскости) расположены на~разных берегах реки
(полоса, ограниченная парой параллельных прямых).
С~помощью циркуля и~линейки определите положение места, в~котором необходимо
построить мост (отрезок, перпендикулярный берегам), так чтобы время пути
из~одного селения в~другое было минимальным.

\item
Некоторую точку плоскости отразили центрально последовательно относительно
вершин треугольника $ABC$ шесть раз в~следующем порядке:
$A$, $B$, $C$, $A$, $B$, $C$.
Докажите, что точка вернулась на~свое исходное место.

\item
Пусть $MN$~--- общая хорда двух окружностей радиуса~$1$.
Луч с~началом на~отрезке~$MN$ перпендикулярный $MN$ пересекает окружности
в~точках $P$ и~$Q$.
Докажите, что $MN^2 + PQ^2 = 4$.

\item
Две окружности радиуса~$1$ касаются друг друга в~точке~$A$.
На~окружностях выбрано по~одной точке $B$ и~$C$ таким образом, что
$\angle BAC = 90^{\circ}$.
Докажите, что $BC = 2$.

\item
На~плоскости даны треугольник, окружность, отрезок длины~$a$ и~прямая~$\ell$.
Циркулем и~линейкой постройте точку~$A$ на~границе треугольника и~точку~$B$
на~окружности, так чтобы длина $AB$ была равна $a$ и~прямые $AB$ и~$\ell$ были
параллельны.

\item
На~медиане~$AM$ треугольника $ABC$ отмечена точка~$X$.
Точка~$Y$ плоскости такова, что $XY \parallel AB$ и~$CY \parallel AM$.
Докажите, что $BX = AY$.

\item
В~выпуклом четырехугольнике $ABCD$ стороны $AB$ и~$CD$ равны.
Докажите, что прямая, соединяющая середины сторон $BC$ и~$AD$ пересекает прямые
$AB$ и~$CD$ под равными углами.

\item
Даны две окружности, лежащие вне друг друга, и~прямая~$\ell$.
Циркулем и~линейкой постройте прямую, параллельную $\ell$, высекающую на~этих
окружностях равные хорды.

\item
Внутри квадрата со~стороной~$1$ лежит фигура (объединение многоугольников)
площади $S$, у~которой нет пары точек на~расстоянии ровно $0{,}001$.
Докажите что
\\
\subproblem $S < 0{,}51$;
\quad
\subproblem $S < 0{,}34$.

\end{problems}

