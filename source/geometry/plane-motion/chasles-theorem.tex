% $date: 2015-12-06
% $timetable:
%   g8r1:
%     2015-12-06:
%       3:

\section*{Теорема Шаля}

% $authors:
% - Андрей Юрьевич Кушнир

Преобразование~$f$ плоскости называется \emph{изометрией} или \emph{движением},
если для любых двух точек плоскости $A$ и~$B$ расстоянием между ними равно
расстоянию между точками $f(A)$ и~$f(B)$.

\claim{Теорема Шаля о~классификации изометрий плоскости}
Любая изометрия плоскости есть параллельный перенос, поворот или скользящая
симметрия.

\begin{problems}

\item
Докажите, что композиция $g \circ f$ двух изометрий $f$ и~$g$~--- также
изометрия.

\item
Докажите, что параллельный перенос, поворот и~осевая симметрия~--- примеры
изометрий.
\\
\emph{Замечание: центральная симметрия~--- это поворот на~$180^{\circ}$.}

\item
Пусть $A$ и~$B$~--- различные точки плоскости, а~$a$ и~$b$~--- неотрицательные
числа.
Докажите, что существует не~более двух точек~$X$ плоскости, таких что $AX = a$
и~$BX = b$.

\item
Пусть $A$, $B$, $C$~--- вершины треугольника, а~$a$, $b$, $c$~---
неотрицательные числа.
Докажите, что существует не~более одной точки~$X$ плоскости, такой что
$AX = a$, $BX = b$, $CX = c$.

\item
Даны два треугольника $ABC$ и~$A'B'C'$, причем
$AB = A'B'$, $BC = B'C'$, $CA = C'A'$.
Докажите, что существует не~более одной изометрии, переводящей
точки $A$, $B$, $C$ в~точки $A'$, $B'$, $C'$ соответственно.

\item
Даны две пары различных точек $A$, $B$ и~$A'$, $B'$, причем $AB = A'B'$.
Докажите, что существует не~более двух изометрий, переводящих
$A$ в~$A'$, а~$B$ в~$B'$.

\item
Направленные отрезки $AB$ и~$A'B'$ равны по~длине и~сонаправлены.
Докажите, что существует параллельный перенос, переводящий
$A$ в~$A'$, а~$B$ в~$B'$.

\item
Направленные отрезки $AB$ и~$A'B'$ равны по~длине, но~не~сонаправлены.
Докажите, что существует поворот, переводящий $A$ в~$A'$, а~$B$ в~$B'$.
\\
\emph{Наводящий вопрос: как построить центр такого поворота?}

\end{problems}

\definition
\emph{Скользящей симметрией} относительно прямой~$\ell$ и~вектора~$\mathbf{v}$,
параллельного $\ell$, называется композиция симметрии относительно
прямой~$\ell$ и~параллельного переноса на~вектор~$\mathbf{v}$.

\begin{problems}

\item
Направленные отрезки $AB$ и~$A'B'$ равны по~длине.
Докажите, что существует скользящая симметрия, переводящая $A$ в~$A'$,
а~$B$ в~$B'$.
\\
\emph{Еще один наводящий вопрос: как построить ось симметрии?}

\item
Пусть $f$~---\enspace
(1) параллельный перенос,
\enspace
(2) поворот
\enspace или\enspace
(3) осевая симметрия.
\\
Рассмотрим напраление (против или по~часовой стрелке), в~котором нужно обойти
вершины $A$, $B$, $C$ треугольника $ABC$, чтобы посетить их в~таком порядке.
Какие из~типов преобразований (1), (2), (3) меняют направление обхода контура
треугольника на~противоположное, а~какие~--- нет?
\\
Эта задача позволит нам поделить все изометрии на~меняющие ориентацию плоскости
и~сохраняющие ориентацию плоскости.

\item
Перечитайте формулировки всех предыдущих задач этого листика еще раз.

\item
Докажите теорему Шаля о~классификации изометрий плоскости.

\end{problems}

