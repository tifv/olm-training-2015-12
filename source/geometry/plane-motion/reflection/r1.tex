% $date: 2015-12-03
% $timetable:
%   g8r1:
%     2015-12-03:
%       2:

\section*{Симметрия}

% $authors:
% - Андрей Юрьевич Кушнир

\begin{problems}

%\item
%Докажите, что четырехугольник, имеющий центр симметрии, является
%параллелограммом.

\item
Докажите, что если фигура имеет две перпендикулярные оси симметрии, то~она
имеет центр симметрии.

\item
Может~ли ограниченная фигура иметь более одного центра симметрии?
А~неограниченная?

%\item
%Дан треугольник $ABC$.
%На~лучах $AB$, $AC$ отложили такие точки $C'$, $B'$, что
%$AB' = AB$, $AC' = AC$.
%Прямая~$B'C'$ пересекла сторону~$BC$ в~точке~$L$.
%Докажите, что $AL$~--- биссектриса треугольника $ABC$.

\item
Внутри угла отмечена точка~$M$.
С~помощью циркуля и~линейки отметьте на~сторонах угла точки $A$ и~$B$ так,
чтобы $M$ была серединой~$AB$.

%\item
%Дан параллелограмм $ABCD$ и~точка~$M$.
%Через точки $A$, $B$, $C$ и~$D$ проведены прямые, параллельные прямым
%$MC$, $MD$, $MA$ и~$MB$ соответственно.
%Докажите, что они пересекаются в~одной точке.

\item
Даны две концентрические окружности.
С~помощью циркуля и~линейки проведите прямую, на~которой эти окружности
высекают три равных отрезка.

\item
В~выпуклом десятиугольнике пары противоположных сторон параллельны и~равны.
Докажите, что его главные диагонали пересекаются в~одной точке.

\item
Пусть $B'$ и~$C'$~--- проекции вершины~$A$ на~биссектрисы углов $B$ и~$C$
треугольника $ABC$ соответственно.
\\
\subproblem
Докажите, что прямая $B'C'$ параллельна $BC$;
\\
\subproblem
докажите, что если $AB' = AC'$, то~$AB = AC$.

\item
В~треугольнике $ABC$ угол~$A$ равен $60^\circ$.
На~лучах $BA$ и~$CA$ отложены отрезки $BX$ и~$CY$, равные стороне~$BC$.
Докажите, что прямая~$XY$ проходит через точку пересечения биссектрис
треугольника $ABC$.

\item
Некоторая окружность пересекает стороны $BC$, $CA$, $AB$ треугольника $ABC$
в~точках $A_1$ и~$A_2$, $B_1$ и~$B_2$, $C_1$ и~$C_2$ соответственно.
Докажите, что если перпендикуляры к~сторонам треугольника, восстановленные
в~точках $A_1$, $B_1$ и~$C_1$, пересекаются в~одной точке, то~и~перпендикуляры
к~сторонам, восстановленные в~точках $A_2$, $B_2$ и~$C_2$, также пересекаются
в~одной точке.

\item
На~диаметре~$AB$ окружности отмечена случайная точка~$X$.
Через точку~$X$ проведена хорда~$CD$, пересекающая хорду~$AB$ под углом
$45^{\circ}$.
Докажите, что величина $CX^2 + DX^2$ не~зависит от~выбора точки~$X$.

\item\emph{Теорема Монжа.}
Докажите, что перпендикуляры, проведенные из~середин сторон или диагонали
вписанного в~окружность четырехугольника к~противоположным сторонам или
другой диагонали соответственно, пересекаются в~одной точке.

\end{problems}

